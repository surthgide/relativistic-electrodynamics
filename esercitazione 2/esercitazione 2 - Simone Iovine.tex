% Esercitazione 2 di Elettrodinamica relativistica di Simone Iovine

\documentclass[12pt,notitlepage]{report}

\usepackage[utf8]{inputenc}
\usepackage[italian]{babel}
\usepackage{datetime}
\usepackage{amsthm,amssymb,amsmath}
\usepackage{mathtools}
\usepackage{cases}
\usepackage{centernot}
\usepackage[makeroom]{cancel}
\usepackage{graphics,graphicx}
\graphicspath{ {images/} }
\usepackage[a4paper,width=170mm,top=25mm,bottom=25mm]{geometry}
\usepackage{setspace}
\usepackage{float}
\usepackage[svgnames]{xcolor}
\usepackage{tikz,pgfplots,tikz-3dplot}
\usepackage{xfrac}
\usepackage{multirow,multicol}
\usepackage{physics}
\usepackage{xcolor}
\usepackage{tcolorbox}
\usepackage{enumitem}
\usepackage{bbm}
\usepackage[toc]{appendix}
\usepackage{parskip}
\usepackage{tikz-cd}
\usepackage{ifthen}
\usepackage{xstring}
\usepackage{witharrows}
\usepackage[scr=boondox, cal=esstix]{mathalpha}
\usepackage{listings}
\usepackage{minitoc}
\usepackage{tensor}

% python code stuff

% check this link for explanation:
% https://www.overleaf.com/learn/latex/Code_Highlighting_with_minted

\definecolor{codegreen}{rgb}{0,0.6,0}
\definecolor{codegray}{rgb}{0.5,0.5,0.5}
\definecolor{codepurple}{rgb}{0.58,0,0.82}
\definecolor{backcolour}{rgb}{0.95,0.95,0.92}

\lstdefinestyle{mystyle}{
	backgroundcolor=\color{backcolour},   
	commentstyle=\color{codegreen},
	keywordstyle=\color{magenta},
	numberstyle=\tiny\color{codegray},
	stringstyle=\color{codepurple},
	basicstyle=\ttfamily\footnotesize,
	breakatwhitespace=false,         
	breaklines=true,                 
	captionpos=b,                    
	keepspaces=true,                 
	numbers=left,                    
	numbersep=5pt,                  
	showspaces=false,                
	showstringspaces=false,
	showtabs=false,                  
	tabsize=2
}

\lstset{style=mystyle}

% i don't know why this new environment doesn't work

%\newcommand{\pycode}[1]{
%	\begin{lstlisting}[language=Python]
%		#1
%	\end{lstlisting}
%}

% symbols definition

\newcommand{\Ld}{\mathcal{L}}
\newcommand{\Hd}{\mathcal{H}}
\renewcommand{\grad}[1]{\va{\boldsymbol{\nabla}}{#1}}
\renewcommand{\curl}[1]{\va{\boldsymbol{\nabla}} \wedge{#1}}
\newcommand{\vas}[1]{\va{\boldsymbol{#1}}}
\newcommand{\bigo}[1]{\mathcal{O} \! \left( #1 \right)}
\newcommand{\smo}[1]{\mathcal{o} \! \left( #1 \right)}
\newcommand{\lnab}[1]{\ln \abs{#1}}

\newcommand{\id}{\operatorname{id}}
\newcommand{\bigone}{\mathbbm{1}}
\newcommand{\sgn}[1]{\operatorname{sgn} \left( #1 \right)}
\newcommand{\diag}[1]{\operatorname{diag} \left( #1 \right)}
\newcommand{\supp}{\operatorname{supp}}
\newcommand{\ob}{\operatorname{Ob}}
\newcommand{\mor}{\operatorname{Mor}}
\newcommand{\st}{\, \middle| \,}
\newcommand{\hatapp}{\; \hat{} \;}
\DeclarePairedDelimiter{\ceil}{\lceil}{\rceil}

%\newcommand{\ie}{i.e. \phantom{$ \!\! $}}

\newcommand{\N}{\mathbb{N}}
\newcommand{\Q}{\mathbb{Q}}
\newcommand{\Z}{\mathbb{Z}}
\newcommand{\R}{\mathbb{R}}
\newcommand{\C}{\mathbb{C}}
\newcommand{\K}{\mathbb{K}}
\newcommand{\T}{\mathbb{T}}
\renewcommand{\S}{\mathbb{S}}

\newcommand{\rp}[1]{\R\mathcal{P}^{#1}}
\newcommand{\B}{\mathcal{B}}

%\newcommand{\g}{\mathfrak{g}}
%\newcommand{\h}{\mathfrak{h}}
%\newcommand{\so}{\mathfrak{so}}
%\newcommand{\su}{\mathfrak{su}}

\newcommand{\notimplies}{\centernot\implies}
\newcommand{\E}{\exists \;}
\newcommand{\ps}{\mathcal{P}}

\newcommand{\hal}{\hspace{13px}}

\DeclareDocumentCommand\dpdv{}{\displaystyle\partialderivative}
\DeclareDocumentCommand\ddv{}{\displaystyle\derivative}

% bracket under equations for explanations

\definecolor{explcol}{RGB}{0, 191, 255} % light blue

\newcommand{\expl}[2]{
	\textcolor{explcol}{ %
		\underbracket[.65pt][4.5pt]{ \textcolor{black}{#2} }_{ \mathclap{ \textcolor{explcol}{#1} } } %
	} %
}

% subequations + align + side arrows environment

% mind this link for explanation of if statement
% https://tex.stackexchange.com/a/314/86

% use this to shift back up the first line
% \vspace{-2.1em}

\makeatletter
\def\subeq{\@ifnextchar[{\@with}{\@without}}
\def\@with[#1]#2{
	\begin{DispWithArrows}[subequations, displaystyle, wrap-lines]
		& \label{#1} \tag{\theequation} \\
		#2
	\end{DispWithArrows}
}
\def\@without#1{
	\begin{DispWithArrows}[subequations, displaystyle, wrap-lines]
		#1
	\end{DispWithArrows}
}
\makeatother

% map definition

\newcommand{\map}[5]{
	\begin{align}
		\begin{split}
			#1 : #2 &\to #3 \\
			#4 &\mapsto #5
		\end{split}
	\end{align}
}

% map w/o equation number

\newcommand{\maps}[5]{
	\begin{align*}
		\begin{split}
			#1 : #2 &\to #3 \\
			#4 &\mapsto #5
		\end{split}
	\end{align*}
}

% image definition

\newcommand{\img}[2]{
	\begin{figure}[H]
		\centering
		\includegraphics[width=#1\textwidth,keepaspectratio]{#2}
	\end{figure}
}

% diagram definition

% mind this link for why & -> \&:
% https://tex.stackexchange.com/questions/15093/single-ampersand-used-with-wrong-catcode-error-using-tikz-matrix-in-beamer

\newcommand{\diagr}[1]{
	\begin{figure}[H]
		\centering
		\begin{tikzcd}[ampersand replacement=\&]
			#1
		\end{tikzcd}
	\end{figure}
}

% side-by-side (two neighbouring boxes)

% \sbs{[width left box]}{[content left box]}{[width right box]}{[content right box]}

\newcommand{\sbs}[4]{
	\noindent\begin{minipage}[c]{#1\textwidth}
		#2
	\end{minipage}
	\begin{minipage}[c]{#3\textwidth}
		#4
	\end{minipage}
}

%

% exercise macro

%\newcounter{solutions}
%\setcounter{solutions}{42} % change this number from 42 to whatever to hide solutions

% \exer{[title]}{[label]}{[exercise text]}{[solution]}

%\newcommand{\exer}[4]{
%	\section{Esercizio #1}\label{#2}
%	
%	\begin{tcolorbox}
%		#3
%	\end{tcolorbox}
%	
%	\ifthenelse{\thesolutions = 42}
%	{#4}
%	{}
%	
%	%
%	
%	\newpage
%}

%

\renewcommand{\contentsname}{Indice}

\makeatletter
\renewcommand{\@chapapp}{Capitolo}
\makeatother

% environments names

\newtheorem{theorem}{Teorema}
\newtheorem{corollary}{Corollario}[theorem]
\newtheorem{lemma}[theorem]{Lemma}
\newtheorem*{remark}{Osservazione}
\newtheorem{definition}{Definizione}[theorem]
\newtheorem{proposition}{Proposizione}[theorem]

\renewcommand*{\proofname}{Dimostrazione}
\renewcommand\qedsymbol{$\square$}

% \highlight[<colour>]{<stuff>}

\newcommand{\highlight}[2][yellow]{\mathchoice%
	{\colorbox{#1}{$\displaystyle#2$}}%
	{\colorbox{#1}{$\textstyle#2$}}%
	{\colorbox{#1}{$\scriptstyle#2$}}%
	{\colorbox{#1}{$\scriptscriptstyle#2$}}}%

% hlc = highlight colour

\definecolor{hlc1}{RGB}{204,241,202} % green
\definecolor{hlc2}{RGB}{255,143,188} % pink
\definecolor{hlc3}{RGB}{163,209,255} % light blue

\newcommand{\hl}[2]{%
	\IfEqCase{#1}{%
		{1}{\highlight[hlc1]{#2}}%
		{2}{\highlight[hlc2]{#2}}%%
		{3}{\highlight[hlc3]{#2}}%
		% can add more cases here as desired
	}[\PackageError{choosecolour}{Undefined option to hl: #1}{}]%
}

% blue links for footsnotes and references

\usepackage[
bookmarksnumbered = true,
linktocpage = true
]{hyperref}

\hypersetup{
	colorlinks = true,
	linkcolor = blue,
	anchorcolor = blue,
	citecolor = blue,
	filecolor = blue,
	urlcolor = blue
}

% table of content stuff

\renewcommand{\thesection}{\arabic{section}}

\makeatletter
\newcommand*{\toccontents}{\@starttoc{toc}}
\makeatother

\begin{document}
	
% titolo

\begin{center}
	{\Large
		\textbf{Esercitazione 2} \\
			Elettrodinamica relativistica \\
			A.A. 2023/2024 \\
			\vspace{0.7cm}
			\textit{Simone Iovine}
		}
\end{center}

\vspace{2cm}

%

\section*{Indice}

\toccontents

%

\section{Esercizio 1}

L'equazione delle onde per onde elettromagnetiche può essere scritta come

\begin{equation}
	\square \varphi \doteq \partial_{\mu} \partial^{\mu} \varphi = 0
\end{equation}

dove $\varphi$ è una funzione scalare e il 4-gradiente è dato da

\begin{equation}
	\partial_{\mu} \doteq \left( \dfrac{1}{c} \pdv{t}, \grad{} \right)
\end{equation}

dunque

\begin{equation}
	\square \doteq \partial_{\mu} \partial^{\mu} = \dfrac{1}{c^{2}} \pdv[2]{t} - \nabla^{2}
\end{equation}

Al fine di verificare l'invarianza di questa equazione, la riscriviamo calcolata in un diverso sistema di riferimento (primato):

\subeq{
	\square' \varphi &\doteq \partial'_{\mu} \partial'^{\mu} \varphi \\
	&= g_{\mu \nu} \partial'^{\nu} \partial'^{\mu} \varphi \\
	&= g_{\mu \nu} \Lambda\indices{^{\nu}_{\rho}} \partial^{\rho} \Lambda\indices{^{\mu}_{\sigma}} \partial^{\sigma} \varphi %
		\Arrow{$\partial^{\mu} \Lambda\indices{^{\nu}_{\rho}} = 0$} \\
	&= g_{\mu \nu} \Lambda\indices{^{\nu}_{\rho}} \Lambda\indices{^{\mu}_{\sigma}} \partial^{\rho} \partial^{\sigma} \varphi
}

dove il tensore metrico per lo spazio piatto è dato da

\begin{equation}
	g_{\mu \nu} = \diag{1, -1, -1, -1}
\end{equation}

Considerando delle trasformazioni di Lorentz, vale la relazione

\begin{equation}
	g_{\mu \nu} (\Lambda_{L})\indices{^{\nu}_{\rho}} (\Lambda_{L})\indices{^{\mu}_{\sigma}} = g_{\rho \sigma}
\end{equation}

dove la matrice di trasformazione è data da

\begin{equation}
	(\Lambda_{L})\indices{^{\mu}_{\nu}} = %
		\bmqty{ \gamma & - \beta^{1} \gamma & - \beta^{2} \gamma & - \beta^{3} \gamma \\
				- \beta^{1} \gamma & 1 + A (\beta^{1})^{2} & A \beta^{1} \beta^{2} & A \beta^{1} \beta^{3} \\
				- \beta^{2} \gamma & A \beta^{2} \beta^{1} & 1 + A (\beta^{2})^{2} & A \beta^{2} \beta^{3} \\
				- \beta^{3} \gamma & A \beta^{3} \beta^{1} & A \beta^{3} \beta^{2} & 1 + A (\beta^{3})^{2}
				} %
	\qquad %
	\begin{cases}
		\vas{\beta} = \dfrac{1}{c} \, (v^{1}, v^{2}, v^{3}) \\\\
		\gamma \doteq \left( 1 - \abs{\vas{\beta}}^{2} \right)^{-1/2} \\\\
		A \doteq \abs{\vas{\beta}}^{-2} (\gamma - 1)
	\end{cases}
\end{equation}              

dunque

\subeq{
	\square' \varphi &= g_{\mu \nu} (\Lambda_{L})\indices{^{\nu}_{\rho}} (\Lambda_{L})\indices{^{\mu}_{\sigma}} \partial^{\rho} \partial^{\sigma} \varphi \\
	&= g_{\rho \sigma} \partial^{\rho} \partial^{\sigma} \varphi \\
	&= \partial_{\rho} \partial^{\rho} \varphi \\
	&= \square \varphi \\
	&= 0
}

Le trasformazioni di Galilei sono invece rappresentate dalla matrice

\begin{equation}
	(\Lambda_{G})\indices{^{\mu}_{\nu}} = %
		\bmqty{ 1 & - v^{1} & - v^{2} & - v^{3} \\
				0 & -1 & 0 & 0 \\
				0 & 0 & -1 & 0 \\
				0 & 0 & 0 & -1
				}
\end{equation}

Per queste trasformazioni di Lorentz quindi la relazione considerata in precedenza produce un risultato diverso

\begin{equation}
	g_{\mu \nu} (\Lambda_{G})\indices{^{\nu}_{\rho}} (\Lambda_{G})\indices{^{\mu}_{\sigma}} = g_{\rho \sigma} + K_{\rho \sigma} %
	\qcomma K_{\rho \sigma} \doteq %
	\bmqty{ 1 & - v^{1} & - v^{2} & - v^{3} \\
			- v^{1} & \left(v^{1}\right)^{2} - 1 & v^{1} v^{2} & v^{1} v^{3} \\
			- v^{2} & v^{2} v^{1} & \left(v^{2}\right)^{2} - 1 & v^{2} v^{3} \\
			- v^{3} & v^{3} v^{1} & v^{3} v^{2} & \left(v^{3}\right)^{2} - 1
			}
\end{equation}

perciò

\subeq{
	\square' \varphi &= g_{\mu \nu} (\Lambda_{G})\indices{^{\nu}_{\rho}} (\Lambda_{G})\indices{^{\mu}_{\sigma}} \partial^{\rho} \partial^{\sigma} \varphi \\
	&= (g_{\rho \sigma} + K_{\rho \sigma}) \partial^{\rho} \partial^{\sigma} \varphi \\
	&= \expl{0}{\square \varphi} + \expl{\text{in generale } \neq 0}{K_{\rho \sigma} \partial^{\rho} \partial^{\sigma} \varphi} \\
	&\neq 0
}

% ==============================================================================

\section{Esercizio 2}

qui

% ==============================================================================

\section{Esercizio 3}

qui

% ==============================================================================

\section{Esercizio 4}

qui

\end{document}
