% Esercitazione 1 di Elettrodinamica relativistica di Simone Iovine

\documentclass[12pt]{report}

\usepackage[utf8]{inputenc}
\usepackage[italian]{babel}
\usepackage{datetime}
\usepackage{amsthm,amssymb,amsmath}
\usepackage{mathtools}
\usepackage{cases}
\usepackage{centernot}
\usepackage[makeroom]{cancel}
\usepackage{graphics,graphicx}
\graphicspath{ {images/} }
\usepackage[a4paper,width=170mm,top=25mm,bottom=25mm]{geometry}
\usepackage{setspace}
\usepackage{float}
\usepackage[svgnames]{xcolor}
\usepackage{tikz,pgfplots,tikz-3dplot}
\usepackage{xfrac}
\usepackage{multirow,multicol}
\usepackage{physics}
\usepackage{xcolor}
\usepackage{tcolorbox}
\usepackage{enumitem}
\usepackage{bbm}
\usepackage[toc]{appendix}
\usepackage{parskip}
\usepackage{tikz-cd}
\usepackage{ifthen}
\usepackage{xstring}
\usepackage{witharrows}
\usepackage[scr=boondox, cal=esstix]{mathalpha}
\usepackage{listings}

% python code stuff

% check this link for explanation:
% https://www.overleaf.com/learn/latex/Code_Highlighting_with_minted

\definecolor{codegreen}{rgb}{0,0.6,0}
\definecolor{codegray}{rgb}{0.5,0.5,0.5}
\definecolor{codepurple}{rgb}{0.58,0,0.82}
\definecolor{backcolour}{rgb}{0.95,0.95,0.92}

\lstdefinestyle{mystyle}{
	backgroundcolor=\color{backcolour},   
	commentstyle=\color{codegreen},
	keywordstyle=\color{magenta},
	numberstyle=\tiny\color{codegray},
	stringstyle=\color{codepurple},
	basicstyle=\ttfamily\footnotesize,
	breakatwhitespace=false,         
	breaklines=true,                 
	captionpos=b,                    
	keepspaces=true,                 
	numbers=left,                    
	numbersep=5pt,                  
	showspaces=false,                
	showstringspaces=false,
	showtabs=false,                  
	tabsize=2
}

\lstset{style=mystyle}

% i don't know why this new environment doesn't work

%\newcommand{\pycode}[1]{
%	\begin{lstlisting}[language=Python]
%		#1
%	\end{lstlisting}
%}

% symbols definition

\newcommand{\Ld}{\mathcal{L}}
\newcommand{\Hd}{\mathcal{H}}
\renewcommand{\grad}[1]{\va{\boldsymbol{\nabla}}{#1}}
\renewcommand{\curl}[1]{\va{\boldsymbol{\nabla}} \wedge{#1}}
\newcommand{\bigo}[1]{\mathcal{O} \! \left( #1 \right)}
\newcommand{\smo}[1]{\mathcal{o} \! \left( #1 \right)}
\newcommand{\lnab}[1]{\ln \abs{#1}}

\newcommand{\id}{\operatorname{id}}
\newcommand{\bigone}{\mathbbm{1}}
\newcommand{\sgn}[1]{\operatorname{sgn} \left( #1 \right)}
\newcommand{\diag}[1]{\operatorname{diag} \left( #1 \right)}
\newcommand{\supp}{\operatorname{supp}}
\newcommand{\ob}{\operatorname{Ob}}
\newcommand{\mor}{\operatorname{Mor}}
\newcommand{\st}{\, \middle| \,}
\newcommand{\hatapp}{\; \hat{} \;}
\DeclarePairedDelimiter{\ceil}{\lceil}{\rceil}

%\newcommand{\ie}{i.e. \phantom{$ \!\! $}}

\newcommand{\N}{\mathbb{N}}
\newcommand{\Q}{\mathbb{Q}}
\newcommand{\Z}{\mathbb{Z}}
\newcommand{\R}{\mathbb{R}}
\newcommand{\C}{\mathbb{C}}
\newcommand{\K}{\mathbb{K}}
\newcommand{\T}{\mathbb{T}}
\renewcommand{\S}{\mathbb{S}}

\newcommand{\rp}[1]{\R\mathcal{P}^{#1}}
\newcommand{\B}{\mathcal{B}}

%\newcommand{\g}{\mathfrak{g}}
%\newcommand{\h}{\mathfrak{h}}
%\newcommand{\so}{\mathfrak{so}}
%\newcommand{\su}{\mathfrak{su}}

\newcommand{\notimplies}{\centernot\implies}
\newcommand{\E}{\exists \;}
\newcommand{\ps}{\mathcal{P}}

\newcommand{\hal}{\hspace{13px}}

\DeclareDocumentCommand\dpdv{}{\displaystyle\partialderivative}
\DeclareDocumentCommand\ddv{}{\displaystyle\derivative}

% bracket under equations for explanations

\definecolor{explcol}{RGB}{0, 191, 255} % light blue

\newcommand{\expl}[2]{
	\textcolor{explcol}{ %
		\underbracket[.65pt][4.5pt]{ \textcolor{black}{#2} }_{ \mathclap{ \textcolor{explcol}{#1} } } %
	} %
}

% subequations + align + side arrows environment

% mind this link for explanation of if statement
% https://tex.stackexchange.com/a/314/86

% use this to shift back up the first line
% \vspace{-2.1em}

\makeatletter
\def\subeq{\@ifnextchar[{\@with}{\@without}}
\def\@with[#1]#2{
	\begin{DispWithArrows}[subequations, displaystyle, wrap-lines]
		& \label{#1} \tag{\theequation} \\
		#2
	\end{DispWithArrows}
}
\def\@without#1{
	\begin{DispWithArrows}[subequations, displaystyle, wrap-lines]
		#1
	\end{DispWithArrows}
}
\makeatother

% map definition

\newcommand{\map}[5]{
	\begin{align}
		\begin{split}
			#1 : #2 &\to #3 \\
			#4 &\mapsto #5
		\end{split}
	\end{align}
}

% map w/o equation number

\newcommand{\maps}[5]{
	\begin{align*}
		\begin{split}
			#1 : #2 &\to #3 \\
			#4 &\mapsto #5
		\end{split}
	\end{align*}
}

% image definition

\newcommand{\img}[2]{
	\begin{figure}[H]
		\centering
		\includegraphics[width=#1\textwidth,keepaspectratio]{#2}
	\end{figure}
}

% diagram definition

% mind this link for why & -> \&:
% https://tex.stackexchange.com/questions/15093/single-ampersand-used-with-wrong-catcode-error-using-tikz-matrix-in-beamer

\newcommand{\diagr}[1]{
	\begin{figure}[H]
		\centering
		\begin{tikzcd}[ampersand replacement=\&]
			#1
		\end{tikzcd}
	\end{figure}
}

% side-by-side (two neighbouring boxes)

% \sbs{[width left box]}{[content left box]}{[width right box]}{[content right box]}

\newcommand{\sbs}[4]{
	\noindent\begin{minipage}[c]{#1\textwidth}
		#2
	\end{minipage}
	\begin{minipage}[c]{#3\textwidth}
		#4
	\end{minipage}
}

%

% exercise macro

\newcounter{solutions}
\setcounter{solutions}{42} % change this number from 42 to whatever to hide solutions

% \exer{[title]}{[label]}{[exercise text]}{[solution]}

\newcommand{\exer}[4]{
	\section{Esercizio #1}\label{#2}
	
	\begin{tcolorbox}
		#3
	\end{tcolorbox}
	
	\ifthenelse{\thesolutions = 42}
	{#4}
	{}
	
	%
	
	\newpage
}

%

\renewcommand{\contentsname}{Indice}

\makeatletter
\renewcommand{\@chapapp}{Capitolo}
\makeatother

% environments names

\newtheorem{theorem}{Teorema}
\newtheorem{corollary}{Corollario}[theorem]
\newtheorem{lemma}[theorem]{Lemma}
\newtheorem*{remark}{Osservazione}
\newtheorem{definition}{Definizione}[theorem]
\newtheorem{proposition}{Proposizione}[theorem]

\renewcommand*{\proofname}{Dimostrazione}
\renewcommand\qedsymbol{$\square$}

% \highlight[<colour>]{<stuff>}

\newcommand{\highlight}[2][yellow]{\mathchoice%
	{\colorbox{#1}{$\displaystyle#2$}}%
	{\colorbox{#1}{$\textstyle#2$}}%
	{\colorbox{#1}{$\scriptstyle#2$}}%
	{\colorbox{#1}{$\scriptscriptstyle#2$}}}%

% hlc = highlight colour

\definecolor{hlc1}{RGB}{204,241,202} % green
\definecolor{hlc2}{RGB}{255,143,188} % pink
\definecolor{hlc3}{RGB}{163,209,255} % light blue

\newcommand{\hl}[2]{%
	\IfEqCase{#1}{%
		{1}{\highlight[hlc1]{#2}}%
		{2}{\highlight[hlc2]{#2}}%%
		{3}{\highlight[hlc3]{#2}}%
		% can add more cases here as desired
	}[\PackageError{choosecolour}{Undefined option to hl: #1}{}]%
}

% blue links for footsnotes and references

\usepackage[
bookmarksnumbered = true,
linktocpage = true
]{hyperref}

\hypersetup{
	colorlinks = true,
	linkcolor = blue,
	anchorcolor = blue,
	citecolor = blue,
	filecolor = blue,
	urlcolor = blue
}

\begin{document}
	
% titolo

\begin{center}
	{\Large
		\textbf{Esercitazione 1} \\
			Elettrodinamica relativistica \\
			A.A. 2023/2024 \\
			\vspace{0.7cm}
			\textit{Simone Iovine}
		}
\end{center}

\vspace{2cm}

%

\section{Esercizio 1}

Riscriviamo la definizione della rapidità

\begin{equation}
	y \doteq \dfrac{1}{2} \ln(\dfrac{E + c p_{z}}{E - c p_{z}}) %
	= \dfrac{1}{2} \ln(\dfrac{1 + \dfrac{v_{z}}{c}}{1 - \dfrac{v_{z}}{c}})
\end{equation}

nel seguente modo

\begin{equation}
	e^{2 y} = \dfrac{E + c p_{z}}{E - c p_{z}} %
	= \dfrac{1 + \dfrac{v_{z}}{c}}{1 - \dfrac{v_{z}}{c}}
\end{equation}

definendo

\begin{equation}
	\beta \doteq \dfrac{u}{c}
\end{equation}

dove $u$ è la velocità del boost, abbiamo anche

\begin{equation}
	e^{2 \eta} = \dfrac{1 + \beta}{1 - \beta}
\end{equation}

Considerando il 4-momento

\begin{equation}
	\va{p} = (E, c \va{p}) %
	= (E, c p_{x}, c p_{y}, c p_{z})
\end{equation}

un boost lungo $z$ può essere calcolato tramite la matrice

\begin{equation}
	\Lambda = \bmqty{\gamma & 0 & 0 & - \beta \gamma \\
					0 & 1 & 0 & 0 \\
					0 & 0 & 1 & 0 \\
					- \beta \gamma & 0 & 0 & \gamma}
\end{equation}

dunque

\begin{equation}
	\va{p}' = \Lambda \va{p}
\end{equation}

Le componenti che hanno subito un cambiamento saranno

\begin{equation}
	\begin{cases}
		E' = \gamma (E - \beta c p_{z}) \\
		c p_{z}' = \gamma (c p_{z} - \beta E)
	\end{cases}
\end{equation}

Siccome i fasci sono incidenti, operiamo il cambiamento di segno $\beta \to - \beta$, da cui

\begin{equation}
	\begin{cases}
		E' = \gamma (E + \beta c p_{z}) \\
		c p_{z}' = \gamma (c p_{z} + \beta E)
	\end{cases}
\end{equation}

A questo punto consideriamo la rapidità nel sistema di riferimento primato

\subeq{
	e^{2 y'} &= \dfrac{E' + c p_{z}'}{E' - c p_{z}'} \\\\
	&= \dfrac{\gamma (E + \beta c p_{z}) + \gamma (c p_{z} + \beta E)}{\gamma (E + \beta c p_{z}) - \gamma (c p_{z} + \beta E)} \\\\
	&= \dfrac{E + \beta c p_{z} + c p_{z} + \beta E}{E + \beta c p_{z} - c p_{z} - \beta E} \\\\
	&= \dfrac{E + \beta c p_{z} + \beta (E + c p_{z})}{E - \beta c p_{z} - \beta (E - c p_{z})} \\\\
	&= \expl{e^{2 y}}{\dfrac{E + c p_{z}}{E - c p_{z}}} \, \expl{e^{2 \eta}}{\dfrac{1 + \beta}{1 - \beta}} \\
	&= e^{2(y + \eta)}
}

\begin{equation}
	e^{2 y'} = e^{2 y'} e^{2 \eta} %
	\implies %
	y'= y + \eta
\end{equation}

Analogamente, si può trovare la stessa soluzione usando la seconda definizione della rapidità. Consideriamo ancora il boost in $z$ della 4-posizione

\begin{equation}
	\begin{cases}
		\va{x} = (c t, \va{x}) \\
		\va{x}' = \Lambda \va{x}
	\end{cases}
\end{equation}

\begin{equation}
	\begin{cases}
		c t' = \gamma (c t - \beta z) \\
		z' = \gamma (z - \beta c t)
	\end{cases}
\end{equation}

da questi possiamo derivare la velocità nel sistema di riferimento primato

\begin{equation}
	\dfrac{v_{z}'}{c} = \dfrac{z'}{c t'} %
	= \dfrac{z - \beta c t}{c t - \beta z} %
	= \dfrac{c t}{c t} \dfrac{\dfrac{z}{ct} - \beta}{1 - \beta \dfrac{z}{ct}} %
	= \dfrac{\dfrac{v_{z}}{c} - \beta}{1 - \beta \dfrac{v_{z}}{c}}
\end{equation}

Ancora una volta, siccome i fasci sono incidenti, operiamo il cambiamento di segno $\beta \to - \beta$, da cui

\begin{equation}
	\dfrac{v_{z}'}{c} = \dfrac{\dfrac{v_{z}}{c} + \beta}{1 + \beta \dfrac{v_{z}}{c}}
\end{equation}

A questo punto sostituiamo nella definizione della rapidità

\subeq{
	e^{2 y'} &= \dfrac{1 + \dfrac{v_{z}'}{c}}{1 - \dfrac{v_{z}'}{c}} \\
	&= \dfrac{1 + \dfrac{\dfrac{v_{z}}{c} + \beta}{1 + \beta \dfrac{v_{z}}{c}}}{1 - \dfrac{\dfrac{v_{z}}{c} + \beta}{1 + \beta \dfrac{v_{z}}{c}}} \\
	&= \dfrac{1 + \beta \dfrac{v_{z}}{c} + \dfrac{v_{z}}{c} + \beta}{1 + \beta \dfrac{v_{z}}{c} - \dfrac{v_{z}}{c} - \beta} \\
	&= \dfrac{1 + \beta + \dfrac{v_{z}}{c} (1 + \beta)}{1 - \beta - \dfrac{v_{z}}{c} (1 - \beta)} \\
	&= \expl{e^{2 y}}{ \dfrac{1 + \dfrac{v_{z}}{c}}{1 - \dfrac{v_{z}}{c}}} \, \expl{e^{2 \eta}}{\dfrac{1 + \beta}{1 - \beta}} \\
	&= e^{2(y + \eta)}
}

\begin{equation}
	e^{2 y'} = e^{2 y'} e^{2 \eta} %
	\implies %
	y'= y + \eta
\end{equation}

\section{Esercizio 2}

Riscriviamo la formula per la sezione d'urto nel seguente modo

\begin{equation}
	\sigma = \dfrac{\dd{\nu}}{\dd{V} \dd{t}} \, \dfrac{1}{v_{rel} \, n_{1} \, n_{2}}
\end{equation}

Tutte le quantità riportate in questa formula sono misurate nel sistema di riferimento del laboratorio, il quale è solidale anche al bersaglio di densità $n_{2}$. \\
Sfruttando i fenomeni di dilatazione del tempo e contrazione delle lunghezze, possiamo riscrivere le quantità riportate nella formula nel sistema di riferimento solidale al fascio incidente di densità $n_{1}$:

\begin{equation}
	\begin{dcases}
		\dd{\nu}' = \dd{\nu} \\
		\dd{V}' = \dfrac{1}{\gamma} \, \dd{V} \\
		\dd{t}' = \gamma \dd{t} \\
		n_{1}' = \dfrac{1}{\gamma} \, n_{1} \\
		n_{2}' = \gamma \, n_{2}
	\end{dcases}
\end{equation}

questo perché:

\begin{itemize}
	\item il numero di urti $\dd{\nu}$ rimane costante indipendentemente dal sistema di riferimento considerato,
	
	\item il volume considerato dove si contano gli urti appare contratto per un osservatore solidale con il fascio in movimento a causa della velocità relativa,
	
	\item per lo stesso motivo, l'intervallo di tempo di misura appare dilatato,
	
	\item la densità di particelle del fascio incidente diminuisce in quanto il volume in cui questa viene calcolata è più grande rispetto al volume misurato nel sistema di riferimento solidale al laboratorio (il quale appare contratto a causa del movimento del fascio incidente rispetto al laboratorio), 
	
	\item la densità di particelle del bersaglio invece aumenta per lo stesso motivo.
\end{itemize}

A questo punto, sostituendo le quantità calcolate nel sistema di riferimento solidale con il fascio incidente nella formula per la sezione d'urto calcolata in questo sistema, otteniamo

\begin{equation}
	\sigma' = \dfrac{\dd{\nu}'}{\dd{V}' \dd{t}'} \, \dfrac{1}{v_{rel} \, n_{1}' \, n_{2}'} %
	= \dfrac{\dd{\nu}}{\dfrac{1}{\gamma} \, \dd{V} \gamma \, \dd{t}'} \, \dfrac{1}{v_{rel} \, \dfrac{1}{\gamma} \, n_{1} \, \gamma \, n_{2}'} %
	= \sigma
\end{equation}

\section{Esercizio 3}

\subsection*{Equazione del moto}

qui

\subsection*{Formula per l'accelerazione}

Poniamo che la velocità $\beta \doteq u / c$ sia limitata all'asse $z$, dunque la trasformazione è data da

\begin{equation}
	\begin{cases}
		c \dd{t}' = \gamma (c \dd{t} - \beta \dd{z}) \\
		\dd{z}' = \gamma (\dd{z} - \beta c \dd{t})
	\end{cases}
\end{equation}

da questi possiamo derivare la velocità $v$ nel sistema di riferimento primato $v'$ come

\begin{equation}
	v' = c \, \dfrac{\dd{z}'}{c \dd{t}'} %
	= c \, \dfrac{\dd{z} - \beta c \dd{t}}{c \dd{t} - \beta \dd{z}} %
	= c \, \dfrac{c \dd{t}}{c \dd{t}} \dfrac{\dfrac{\dd{z}}{c \dd{t}} - \beta}{1 - \beta \dfrac{\dd{z}}{c \dd{t}}} %
	= c \, \dfrac{\dfrac{v}{c} - \beta}{1 - \beta \dfrac{v}{c}} %
	= \dfrac{v - \beta c}{1 - \beta \dfrac{v}{c}}
\end{equation}

Usando

\begin{equation}
	\gamma \doteq \dfrac{1}{\sqrt{1 - \beta^{2}}} %
	\implies %
	1 - \beta^{2} = \dfrac{1}{\gamma^{2}}
\end{equation}

calcoliamo il differenziale della velocità

\subeq{
	\dd{v}' &= \dd{(v - \beta c)} \dfrac{1}{1 - \beta \dfrac{v}{c}} + (v - \beta c) \dd{\left( \dfrac{1}{1 - \beta \dfrac{v}{c}} \right)} \\
	&= \dfrac{\dd{v}}{1 - \beta \dfrac{v}{c}} + (v - \beta c) \left( - \dfrac{1}{\left(1 - \beta \dfrac{v}{c}\right)^{2}} \right) \left( - \dfrac{\beta}{c} \dd{v} \right) \\
	&= \left( \dfrac{1}{1 - \beta \dfrac{v}{c}} + \dfrac{\left( \dfrac{v}{c} - \beta \right) \beta}{\left(1 - \beta \dfrac{v}{c}\right)^{2}} \right) \dd{v} \\
	&= \dfrac{1 - \beta \dfrac{v}{c} + \beta \dfrac{v}{c} - \beta^{2}}{\left(1 - \beta \dfrac{v}{c}\right)^{2}} \dd{v} \\
	&= \dfrac{1}{\gamma^{2} \left(1 - \beta \dfrac{v}{c}\right)^{2}} \dd{v}
}

A questo punto possiamo calcolare l'accelerazione

\subeq{
	a' &\doteq \dv{v'}{t'} \\
	&= \dfrac{\dd{v}}{\gamma^{2} \left(1 - \beta \dfrac{v}{c}\right)^{2}} \dfrac{c}{\gamma (c \dd{t} - \beta \dd{z})} \\
	&= \dfrac{c}{\gamma^{3} \left(1 - \beta \dfrac{v}{c}\right)^{2} (c \dd{t} - \beta \dd{z})} \dd{v} \\
	&= \dfrac{c}{\gamma^{3} \left(1 - \beta \dfrac{v}{c}\right)^{2} c \dd{t} \left(1 - \beta \expl{v/c}{\dfrac{\dd{z}}{c \dd{t}}}\right)} \dd{v} \\
	&= \dfrac{1}{\gamma^{3} \left(1 - \beta \dfrac{v}{c}\right)^{3}} \dv{v}{t} \\
	&= \dfrac{1}{\gamma^{3} \left(1 - \beta \dfrac{v}{c}\right)^{3}} \, a
}

Essendo la velocità del boost la stessa della velocità di cui abbiamo operato la trasformazione, i.e. $\beta = v/c$, otteniamo

\begin{equation}
	a' = \gamma^{3} a
\end{equation}

\subsection*{Accelerazione di un protone}

qui






\section{Esercizio 4}

i

\end{document}
